\subsection{Data Preprocessing}
\label{subsec:preprocessing}

This section outlines the preprocessing steps applied to the Hepatitis and Mushroom datasets.

\subsubsection{Dealing with Different Ranges and different types}
To manage the varying types and ranges of attributes in our datasets, we implemented specific preprocessing techniques.
For the nominal attributes in both the Mushroom and Hepatitis datasets, we used label encoding.
This technique converts categorical values into numerical labels, enabling the algorithms to interpret the data correctly.
While we considered one-hot encoding to avoid implying any ordinal relationship among categories, we opted for label encoding due to its simplicity and reduced dimensionality, as one-hot encoding would significantly increase dimensions and lead to a sparse space, making accurate predictions more challenging, particularly for KNN with the Mushroom dataset's numerous nominal features.

For the numerical attributes in the Hepatitis dataset, we applied min-max scaling to rescale the data to a fixed range of [0, 1]. This normalization is crucial for distance-based algorithms like KNN and SVM, ensuring all features contribute equally to model performance.
We evaluated other scaling methods, such as standardization, but chose min-max scaling for its effectiveness in maintaining the original data distribution \cite{data_cleaning}.

\subsubsection{Dealing with Missing Values}
Addressing missing values is a critical step in preparing our datasets for analysis, as they can significantly impact model performance.
In the case of the nominal attributes in both the Mushroom and Hepatitis datasets, we opted to impute missing values with the majority class.
This method is straightforward and effective for maintaining dataset integrity.
However, it can also introduce bias, particularly in the Hepatitis dataset, where the majority class represents 79.35\% of instances.
Relying on this method may lead to a situation where the imputed values disproportionately favor the majority class, thereby affecting the overall distribution
and potentially skewing the results \cite{data_cleaning}.

For the numerical attributes in the Hepatitis dataset, we used the mean of the available data to fill in missing values.
This approach preserves the overall data distribution and is easy to implement, but it is not without its drawbacks.
The mean can be heavily influenced by outliers, which might distort the data and lead to less accurate predictions.
This is especially important in medical datasets, where extreme values may carry significant meaning.

We also considered employing K-Nearest Neighbors (KNN) for imputing missing values, as it could provide a more nuanced approach by considering the nearest data points for each instance.
However, we ultimately decided against this option to avoid introducing bias into our evaluation.
Since KNN is one of the algorithms we are testing, using it for imputation could influence its performance and lead to skewed results.
Therefore, we chose the more straightforward methods of majority class imputation for nominal values and mean imputation for numerical values, allowing for a clearer assessment of the models’ effectiveness without confounding factors.


