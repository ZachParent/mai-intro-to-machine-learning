\section{Introduction}
\label{sec:introduction}

This is the introduction section of the report. It provides background information and context for the study, states the research question or hypothesis, and briefly outlines the approach taken.
A report structure overview is also provided to guide the reader through the document.

\subsection{Background Information}

Classification in the context of machine learning is the process of predicting the class or category of a given data point based on its features.
For example: given a set of emails, classify each email as spam or not spam. 
It is a supervised learning technique that is used to assign labels to data points based on their characteristics.
Classification is an ideal technique for the required analysis as outlined in this report.\par

K-Nearest Neighbour (KNN) is a supervised machine learning algorithm that has been chosen as the primary method for this study.
It is a simple yet powerful algorithm that classifies new cases based on similarity to existing data points.
While KNN can be used for both classification and regression problems, our focus is on classification due to the nature of the datasets used.\par

This report details the analysis of two datasets: `hepatitis' and `mushroom'. These are two of many datasets provided to us as part of this assignment.
We decided upon these two datasets due to the stark contrast between them in terms of complexity and size.

\subsection{Research Question or Hypothesis}
The primary research question addressed in this report is: `Can the k-Nearest Neighbour algorithm effectively classify
data points in the hepatitis and mushroom datasets with high accuracy'?.\par

The specific objectives of this study are:
\begin{enumerate}
    \item To evaluate KNN's classification performance on datasets of varying sizes and complexities
    \item To determine optimal KNN hyperparameter settings for each dataset
    \item To assess the algorithm's robustness and limitations in different scenarios
\end{enumerate}

We hypothesize that KNN will perform excellently on the mushroom dataset due to the simplicity of the data,
but may struggle with the hepatitis dataset due to the complexity of the data in that dataset.

% TODO: Discuss adding additional steps for SVM and reduction techniques
\subsection{Approach and Methodology}
The approach taken in this report involves the following steps:
\begin{enumerate}
    \item Data Preprocessing: Both datasets are preprocessed, ensuring the data is clean and ready for analysis.
    \item Model Training: Different combinations of hyperparameters are used to train the KNN algorithm on the training dataset.
    \item Model Evaluation: The performance of the best KNN algorithm from the previous step is evaluated using the test dataset.
\end{enumerate}

\subsection{Report Structure}
The report is structured as follows:
\begin{enumerate}
    \item Introduction (Section 1): Provides background information, states the research question, and outlines the approach.
    \item Data (Section 2): Describes the data used in the study, along with the source, characteristics, relevance, and the preprocessing techniques that were applied.
    \item Methods (Section 3): Describes the methodology used in the study, including the algorithms, techniques, and tools used.
    \item Results and Analysis (Section 4): Findings of the study are presented, including relevant data, statistics, and figures to help visualise the data.
    \item Conclusion (Section 5): Summarizes main findings of the study and their significance.
\end{enumerate}