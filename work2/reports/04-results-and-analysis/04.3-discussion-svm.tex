\subsection{Support Vector Machines (SVM) Analysis}
\label{subsubsec:discussion-svm}

A key advantage of the SVM over KNN is that SVMs are much faster during consultation time. As seen in \autoref{fig:model_comparison_mushroom},
SVMs tend to have much lower test times than KNN. This is especially noticable for large datasets like the mushroom dataset.

\begin{figure}
    \centering
    \includegraphics[width=0.9\textwidth]{figures/model_comparison_mushroom.png}
    \caption{SVM and KNN model comparison on Mushroom Dataset}
    \label{fig:model_comparison_mushroom}
\end{figure}

\subsubsection{Mushroom Dataset Performance}

\begin{description}
    \item[\textbf{Best Configuration:}]\leavevmode
        \begin{itemize}
            \item C = 5 and C = 50 with Polynomial Kernel
            \item C = 50 with RBF Kernel
            \item \textbf{Results:} F1 score = 1.000, Train Time = 0.117225s, Test Time = 0.003384s
        \end{itemize}
    
    \item[\textbf{Performance Range:}]\leavevmode
        \begin{itemize}
            \item Top configurations achieved perfect F1 scores (1.000)
            \item Test times varied from 0.002183s to 0.015565s
        \end{itemize}
\end{description}

\subsubsection*{Observations}
The results for the mushroom dataset show that the SVM model performed consistently strong across all configurations,
albeit some performed better than others. The best performing model used a Polynomial kernel with $C=5$ or $C=50$
or an RBF kernel with $C=50$, with all achieving a perfect mean F1 score of 100\%.
This result was closely followed by the RBF kernel with $C=5$ achieving a mean F1 score of 99.96\%.

\begin{figure}
    \centering
    \includegraphics[width=0.9\textwidth]{figures/ranked_folds_SVM_mushroom.png}
    \caption{SVM Ranked Folds for Mushroom Dataset}
    \label{fig:ranked_folds_SVM_mushroom}
\end{figure}

The above figure shows the ranked folds for the SVM algorithm on the mushroom dataset.
The figure illustrates the variation in performance across different SVM models, with some models performing 
better than others. Notably, the presence of outliers (shown as points) in several models indicates
occasional deviations from the mean performance. The median ranks vary considerably across the different models,
with only C5Lin and C7Lin showing consistent performance across all folds.

The mushroom dataset has some especially obvious cases, in which the $C=5$ and $C=50$ polynomial models
outperformed all other models across every fold. The $C=50$ RBF model was also among the best performing models overall. (See \autoref{tab:svm_results_mushroom}).

\subsubsection{Hepatitis Dataset Performance}

\begin{description}
    \item[\textbf{Best Configuration:}]\leavevmode
        \begin{itemize}
            \item C = 50 with RBF Kernel
            \item \textbf{Results:} F1 score = 0.972671, Train Time = 0.000885s, Test Time = 0.000371s
        \end{itemize}
    
    \item[\textbf{Performance Range:}]\leavevmode
        \begin{itemize}
            \item Best configuration: F1 = 0.972671 (C=50, RBF)
            \item Next best configurations: F1 = 0.970963 (C=5, RBF)
        \end{itemize}
\end{description}

\subsubsection*{Observations}
The results for the hepatitis dataset show that the SVM model also performed consistently strong across all configurations.
(See \autoref{tab:svm_results_mushroom}).
The best performing model used an RBF kernel with $C=50$.
This SVM model achieved a mean F1 score of 97.3\%.

\begin{figure}
    \centering
    \includegraphics[width=0.9\textwidth]{figures/ranked_folds_SVM_hepatitis.png}
    \caption{SVM Ranked Folds for Hepatitis Dataset}
    \label{fig:ranked_folds_SVM_hepatitis}
\end{figure}

The above figure shows the ranked folds for the SVM algorithm on the mushroom dataset.
The figure illustrates the variation in performance across the different SVM models and folds.
When considering the mean, SVM models `C3RBF' and `C5RBF' performed the best across all folds.

The hepatitis dataset does not have as extreme of a case, but there are still some models that stand out.
The linear kernel models were consistently worse than the $C=50$ and $C=5$ RBF models.

\subsubsection*{Overall Comparison}
A higher C value tends to perform better for both datasets, suggesting that greater regularization is better for both datasets.
However, the best performing models for each dataset used different kernels paired with these C values
(poly for mushroom and rbf for hepatitis). See \autoref{tab:svm_results_mushroom} and \autoref{tab:svm_results_hepatitis}.

\subsection*{Support Vector Machines (SVM) Statistical Analysis}
To compare the performance across model configurations, we employed statistical analysis methods
(see \autoref{sec:statistical-analysis}) to determine whether the various configurations showed
statistically significant differences in performance.

As seen in \autoref{fig:ranked_folds_SVM_hepatitis} and \autoref{fig:ranked_folds_SVM_mushroom},
some models achieved better results than others. By comparing the ranks among each of the 10 folds,
we can check to see if there were any cases in which 1 model always outperformed another.

The mushroom dataset has some especially obvious cases, in which the $C=5$ and $C=50$ polynomial models
outperformed all other models across every fold. The $C=50$ RBF model was also among the best performing models overall.

The hepatitis dataset does not have as extreme of a case, but there are still some models that stand out.
The linear kernel models were consistently worse than the $C=50$ and $C=5$ RBF models.

\begin{figure}
    \centering
    \includegraphics[width=0.4\textwidth]{figures/nemenyi_test_results_SVM_hepatitis.png}
    \caption{Nemenyi Test Results for SVM on Hepatitis Dataset}
    \label{fig:nemenyi_test_results_SVM_hepatitis}
\end{figure}

\begin{figure}
    \centering
    \includegraphics[width=0.4\textwidth]{figures/nemenyi_test_results_SVM_mushroom.png}
    \caption{Nemenyi Test Results for SVM on Mushroom Dataset}
    \label{fig:nemenyi_test_results_SVM_mushroom}
\end{figure}

To identify significant pairs at a glance, we performed the Nemenyi test on all model pairs,
and present them in heatmaps (see \autoref{fig:nemenyi_test_results_SVM_hepatitis} and \autoref{fig:nemenyi_test_results_SVM_mushroom}).
The cells represent the p-values for each pair of models. A value of 0.05 or lower indicates that the models are statistically different
at the 95\% confidence level. The full tables containing the p-values for each pair of significant models are provided in \autoref{tab:svm_significant_pairs_hepatitis} and \autoref{tab:svm_significant_pairs_mushroom}.

\begin{table}
\centering
\caption{Significant Differences in SVM Models}
\label{tab:svm_significant_pairs_hepatitis}
\begin{tabular}{rlrrrrrrrrrrrrrrrrrrrrrrrrrrrrrrrrrl}
\toprule
C & kernel_type & f1_0 & f1_1 & f1_2 & f1_3 & f1_4 & f1_5 & f1_6 & f1_7 & f1_8 & f1_9 & train_time_0 & train_time_1 & train_time_2 & train_time_3 & train_time_4 & train_time_5 & train_time_6 & train_time_7 & train_time_8 & train_time_9 & test_time_0 & test_time_1 & test_time_2 & test_time_3 & test_time_4 & test_time_5 & test_time_6 & test_time_7 & test_time_8 & test_time_9 & mean_f1 & mean_train_time & mean_test_time & model_label \\
\midrule
3 & rbf & 0.857 & 0.966 & 1.000 & 1.000 & 1.000 & 1.000 & 0.909 & 1.000 & 0.957 & 1.000 & 0.001 & 0.001 & 0.001 & 0.001 & 0.001 & 0.001 & 0.001 & 0.001 & 0.001 & 0.001 & 0.000 & 0.000 & 0.000 & 0.000 & 0.000 & 0.000 & 0.000 & 0.000 & 0.000 & 0.000 & 0.969 & 0.001 & 0.000 & C3Rbf \\
5 & rbf & 0.828 & 0.966 & 1.000 & 1.000 & 0.960 & 1.000 & 0.957 & 1.000 & 1.000 & 1.000 & 0.001 & 0.001 & 0.001 & 0.001 & 0.001 & 0.001 & 0.001 & 0.001 & 0.001 & 0.001 & 0.000 & 0.000 & 0.000 & 0.000 & 0.000 & 0.000 & 0.000 & 0.000 & 0.000 & 0.000 & 0.971 & 0.001 & 0.000 & C5Rbf \\
7 & sigmoid & 0.857 & 0.667 & 0.815 & 0.909 & 0.846 & 0.846 & 0.846 & 0.750 & 0.909 & 0.846 & 0.001 & 0.001 & 0.001 & 0.001 & 0.001 & 0.001 & 0.001 & 0.001 & 0.001 & 0.001 & 0.000 & 0.000 & 0.000 & 0.000 & 0.000 & 0.000 & 0.000 & 0.000 & 0.000 & 0.000 & 0.829 & 0.001 & 0.000 & C7Sig \\
\bottomrule
\end{tabular}
\end{table}

\begin{table}
\centering
\caption{Significant Differences in SVM Models}
\label{tab:svm_significant_pairs_mushroom}
\begin{tabular}{rlr}
\toprule
C & Kernel Type & Mean F1 \\
\midrule
1 & linear & 0.975 \\
1 & rbf & 0.989 \\
3 & linear & 0.980 \\
3 & rbf & 0.999 \\
5 & linear & 0.983 \\
5 & rbf & 1.000 \\
7 & linear & 0.983 \\
7 & sigmoid & 0.436 \\
\bottomrule
\end{tabular}
\end{table}


\begin{figure}
    \centering
    \includegraphics[width=0.8\textwidth]{figures/interaction_effects_SVM_hepatitis.png}
    \caption{Interaction Effects for SVM on Hepatitis Dataset}
    \label{fig:interaction_effects_SVM_hepatitis}
\end{figure}

\begin{figure}
    \centering
    \includegraphics[width=0.8\textwidth]{figures/interaction_effects_SVM_mushroom.png}
    \caption{Interaction Effects for SVM on Mushroom Dataset}
    \label{fig:interaction_effects_SVM_mushroom}
\end{figure}

To further analyze the hyperparameters we plotted the effects of each hyperparameter on the F1 score.

In \autoref{fig:interaction_effects_SVM_hepatitis} and \autoref{fig:interaction_effects_SVM_mushroom} we see that the Kernel Type is a stronger predictor of performance than the $C$ parameter, and that
the best performing models tend to use the rbf or poly kernels, and that higher values of $C$ tend to perform worse.

\paragraph{Hepatitis Dataset}
The best performing model was an RBF kernel with $C=50$, achieving a mean F1 score of 97.3\%, as seen in \autoref{tab:svm_results_hepatitis}.

\paragraph{Mushroom Dataset}
The best performing models were the polynomial kernel with $C=5$ and $C=50$, and RBF kernel with $C=50$, all achieving a perfect mean F1 score of 100\%, as seen in \autoref{tab:svm_results_mushroom}.

