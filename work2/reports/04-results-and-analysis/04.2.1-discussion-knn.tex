\subsubsection{k-Nearest Neighbors (kNN) Analysis}
\label{subsubsec:discussion-knn}

This section interprets the results of the kNN algorithm, discussing its performance and implications.

The kNN algorithm was evaluated using several performance metrics, including accuracy, precision, and recall.

% TODO: Summarize the performance metrics of kNN (e.g., accuracy, precision, recall)




\paragraph{Impact of Different k Values}
% The choice of the parameter \( k \) significantly impacts the performance of the kNN algorithm.
% Figure \ref{fig:k-values} illustrates how accuracy varies with different values of \( k \).
% It was observed that the optimal value of \( k \) was 5, beyond which the performance started to degrade.

% \begin{figure}[h]
%     \centering
%     %\includegraphics[width=0.8\textwidth]{path/to/knn-k-values-plot.png}
%     \caption{Impact of different \( k \) values on kNN performance}
%     \label{fig:k-values}
% \end{figure}

\paragraph{Unexpected Findings and Limitations}
One unexpected finding was that the kNN algorithm performed poorly on imbalanced datasets.
This limitation suggests that further preprocessing, such as oversampling or undersampling,
might be necessary to improve performance.

% Add more content here as needed
% TODO: Compare kNN results with other methods used in the study
% TODO: Discuss the impact of different k values on the results
% TODO: Address any unexpected findings or limitations specific to kNN

% Add more content here as needed
