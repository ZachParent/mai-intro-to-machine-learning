\subsubsection{Support Vector Machines (SVM) Analysis}
\label{subsubsec:discussion-svm}

% SVMs performed well on both the hepatitis and the mushroom datasets,
% achieving peak accuracy and F1 scores which match the best results from KNN.

A key advantage of the SVM over KNN is that SVMs are much faster during consultation time. As seen in \autoref{fig:model_comparison_mushroom},
SVMs tend to have much lower test times than KNN. This is expecially noticable for large datasets like the mushroom dataset.

\begin{figure}
    \centering
    \includegraphics[width=0.9\textwidth]{figures/model_comparison_mushroom.png}
    \caption{SVM and KNN Comparison}
    \label{fig:model_comparison_mushroom}
\end{figure}


\subsubsection*{Hyperparameter Comparison}

For the hepatitis dataset, the configuration using the \textbf{RBF} kernel and $C=7$
achieved outstanding accuracy and F1 scores. However, because of the simple predictability
of the mushroom data, the simple \textbf{Polynomial} kernel performed better when paired with $C=1$.
(See \autoref{tab:svm_results_hepatitis} and \autoref{tab:svm_results_mushroom}).

To compare the performance across model configurations, we employed statistical analysis methods
(see \autoref{sec:statistical-analysis}) to determine whether the various configurations showed
statistically significant differences in performance.

\begin{figure}
    \centering
    \includegraphics[width=0.9\textwidth]{figures/ranked_folds_SVM_hepatitis.png}
    \caption{SVM Ranked Folds for Hepatitis Dataset}
    \label{fig:ranked_folds_svm_hepatitis}
\end{figure}

\begin{figure}
    \centering
    \includegraphics[width=0.9\textwidth]{figures/ranked_folds_SVM_mushroom.png}
    \caption{SVM Ranked Folds for Mushroom Dataset}
    \label{fig:ranked_folds_svm_mushroom}
\end{figure}

As seen in \autoref{fig:ranked_folds_svm_hepatitis} and \autoref{fig:ranked_folds_svm_mushroom},
some models achieved significantly better results than others. By comparing the ranks among each of the 10 folds,
we can check to see if there were any cases in which 1 model always outperformed another.

The mushroom dataset has some especially obvious cases, in which the $C=3$ and $C=5$ RBF models
outperformed all other models across every fold. The $C=5$ RBF model was the best performing model overall.

The hepatitis dataset does not have as extreme of a case, but there are still some models that stand out.
The $C=7$ Sigmoid model was consistently worse than the $C=3$ and $C=5$ RBF models.

\begin{figure}
    \centering
    \includegraphics[width=0.4\textwidth]{figures/nemenyi_test_results_SVM_hepatitis.png}
    \caption{Nemenyi Test Results for SVM on Hepatitis Dataset}
    \label{fig:nemenyi_test_results_SVM_hepatitis}
\end{figure}

\begin{figure}
    \centering
    \includegraphics[width=0.4\textwidth]{figures/nemenyi_test_results_SVM_mushroom.png}
    \caption{Nemenyi Test Results for SVM on Mushroom Dataset}
    \label{fig:nemenyi_test_results_SVM_mushroom}
\end{figure}

To identify significant pairs at a glance, we performed the Nemenyi test on all model pairs,
and present them in heatmaps (see \autoref{fig:nemenyi_test_results_SVM_hepatitis} and \autoref{fig:nemenyi_test_results_SVM_mushroom}).
The cells represent the p-values for each pair of models. A value of 0.05 or lower indicates that the models are statistically different
at the 95\% confidence level. The full tables containing the p-values for each pair of significant models are provided in \autoref{tab:svm_significant_pairs_hepatitis} and \autoref{tab:svm_significant_pairs_mushroom}.

\begin{table}
\centering
\caption{Significant Differences in SVM Models}
\label{tab:svm_significant_pairs_hepatitis}
\begin{tabular}{rlrrrrrrrrrrrrrrrrrrrrrrrrrrrrrrrrrl}
\toprule
C & kernel_type & f1_0 & f1_1 & f1_2 & f1_3 & f1_4 & f1_5 & f1_6 & f1_7 & f1_8 & f1_9 & train_time_0 & train_time_1 & train_time_2 & train_time_3 & train_time_4 & train_time_5 & train_time_6 & train_time_7 & train_time_8 & train_time_9 & test_time_0 & test_time_1 & test_time_2 & test_time_3 & test_time_4 & test_time_5 & test_time_6 & test_time_7 & test_time_8 & test_time_9 & mean_f1 & mean_train_time & mean_test_time & model_label \\
\midrule
3 & rbf & 0.857 & 0.966 & 1.000 & 1.000 & 1.000 & 1.000 & 0.909 & 1.000 & 0.957 & 1.000 & 0.001 & 0.001 & 0.001 & 0.001 & 0.001 & 0.001 & 0.001 & 0.001 & 0.001 & 0.001 & 0.000 & 0.000 & 0.000 & 0.000 & 0.000 & 0.000 & 0.000 & 0.000 & 0.000 & 0.000 & 0.969 & 0.001 & 0.000 & C3Rbf \\
5 & rbf & 0.828 & 0.966 & 1.000 & 1.000 & 0.960 & 1.000 & 0.957 & 1.000 & 1.000 & 1.000 & 0.001 & 0.001 & 0.001 & 0.001 & 0.001 & 0.001 & 0.001 & 0.001 & 0.001 & 0.001 & 0.000 & 0.000 & 0.000 & 0.000 & 0.000 & 0.000 & 0.000 & 0.000 & 0.000 & 0.000 & 0.971 & 0.001 & 0.000 & C5Rbf \\
7 & sigmoid & 0.857 & 0.667 & 0.815 & 0.909 & 0.846 & 0.846 & 0.846 & 0.750 & 0.909 & 0.846 & 0.001 & 0.001 & 0.001 & 0.001 & 0.001 & 0.001 & 0.001 & 0.001 & 0.001 & 0.001 & 0.000 & 0.000 & 0.000 & 0.000 & 0.000 & 0.000 & 0.000 & 0.000 & 0.000 & 0.000 & 0.829 & 0.001 & 0.000 & C7Sig \\
\bottomrule
\end{tabular}
\end{table}

\begin{table}
\centering
\caption{Significant Differences in SVM Models}
\label{tab:svm_significant_pairs_mushroom}
\begin{tabular}{rlr}
\toprule
C & Kernel Type & Mean F1 \\
\midrule
1 & linear & 0.975 \\
1 & rbf & 0.989 \\
3 & linear & 0.980 \\
3 & rbf & 0.999 \\
5 & linear & 0.983 \\
5 & rbf & 1.000 \\
7 & linear & 0.983 \\
7 & sigmoid & 0.436 \\
\bottomrule
\end{tabular}
\end{table}

