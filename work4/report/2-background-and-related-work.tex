\section{Background and Related Work}
\label{sec:background-and-related-work}

\paragraph{}
Our previous work \cite{our-clustering-work} established a comprehensive comparison 
of clustering algorithms, with particular focus on K-Means variants and density-based 
approaches. We demonstrated that while Global K-Means \cite{global_kmeans} offers 
robust performance across different datasets, density-based methods like OPTICS 
\cite{1999-optics} excel at identifying clusters with complex shapes. This foundation 
motivates our current investigation into how dimensionality reduction affects these 
clustering approaches.

\paragraph{}
The challenge of high-dimensional data analysis, often referred to as the "curse of 
dimensionality" \cite{dimensionality}, presents significant obstacles for clustering 
algorithms. As the number of dimensions increases, the data becomes increasingly sparse, 
making it difficult to identify meaningful patterns and clusters. Dimensionality 
reduction techniques, particularly Principal Component Analysis (PCA), address this 
challenge by transforming the data into a lower-dimensional space while preserving 
its essential characteristics. This transformation not only makes clustering more 
computationally efficient but can also improve the quality of the resulting clusters 
by focusing on the most informative features of the data.


