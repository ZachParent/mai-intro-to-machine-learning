\section{Conclusion}
\label{sec:conclusion}

Clustering is a core machine learning technique used in many applications, but choosing the right clustering algorithm and hyperparameters is not always straightforward. In this report, we have evaluated several clustering algorithms and their hyperparameters on three distinct datasets, exploring the strengths and weaknesses of each method. Here we present our key findings and practical implications of the results.

\subsection{Key Findings}

\subsection{Practical Implications}

\subsection{Limitations and Future Work}


\subsection{Final Remarks}

Each clustering algorithm demonstrates distinct strengths and weaknesses, suggesting that the choice of algorithm should be primarily driven by specific application requirements and data characteristics. While K-means and Fuzzy C-means offer good general-purpose solutions, specialized algorithms like OPTICS and Spectral clustering provide advantages for specific data distributions. Future work should focus on developing hybrid approaches that can combine the strengths of multiple algorithms while mitigating their individual weaknesses.

